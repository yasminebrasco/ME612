\documentclass[a4paper]{article}

%% Language and font encodings
\usepackage[portuguese]{babel}
\usepackage[utf8x]{inputenc}
\usepackage[T1]{fontenc}

%% Sets page size and margins
\usepackage[a4paper,top=3cm,bottom=2cm,left=3cm,right=2cm]{geometry}
\usepackage{setspace}

%% Useful packages
\usepackage{amsmath}
\usepackage{graphicx}
\usepackage[colorinlistoftodos]{todonotes}
\usepackage[colorlinks=true, allcolors=blue]{hyperref}
\usepackage{indentfirst}


\begin{document}
\fontfamily{ptm}

\begin{titlepage}
\begin{center}
\includegraphics[scale=0.3]{Logo_Unicamp.jpg}
\textbf{
	\begin{large}
		\\Universidade Estadual de Campinas\\
		Métodos Estatísticos Aplicados à Ciências Biológicas\\\vspace{5cm}
	\end{large}
	\begin{Huge}
		Modelos de Fragilidade Gama e Lognormal\\\vspace{6cm}
	\end{Huge}
	\begin{large}
	Gabrielle Cristina \\
	Karina Hernandes \\
	Melissa Palermo \\
	Yasmine Brasco 148255\\\vspace{5cm}
	\end{large}
}
	\begin{small}
		Campinas - SP\\
		2016
	\end{small}
\end{center}
\end{titlepage}


\setstretch{1.5}
\section{Introdução}

Your introduction goes here! Some examples of commonly used commands and features are listed below, to help you get started. If you have a question, please use the help menu (``?'') on the top bar to search for help or ask us a question. 

\section{Objetivos}

\section{Modelos de Fragilidade}

\subsection{Conceitos}

First you have to upload the image file from your computer using the upload link the project menu. Then use the includegraphics command to include it in your document. Use the figure environment and the caption command to add a number and a caption to your figure. See the code for Figure in this section for an example.

\subsection{Gama}

Graças a sua conveniência algébrica, a distribuição gama é muito utilizada para modelar variáveis de fragilidade. Seja $Z_{t}$ tal variável, define-se que $E\left(Z_{t}\right)=1$ e $Var\left(Z_{t}\right)=\xi$. Logo $Z_{t}\sim\Gamma\left(\frac{1}{\xi},\frac{1}{\xi}\right)$ com função densidade de probabilidade dada por:

\begin{center}
$f_{Z}\left(z\right)=\frac{\left(\frac{1}{\xi}\right)^{\frac{1}{\xi}}}{{\Gamma{\left(\frac{1}{\xi}\right)}}}z^{\frac{1}{\xi}-1}e^{-\frac{z}{\xi}}$
\end{center}

O modelo de fragilidade Gama é definido como:

\begin{center}
$\lambda_{ij}\left(t\right)=z_{j}\lambda_{0}\left(t\right)e^{x'_{ij}\beta}$
\end{center}

\subsubsection{Parâmetros}

Comments can be added to your project by clicking on the comment icon in the toolbar above.
To reply to a comment, simply click the reply button in the lower right corner of the comment, and you can close them when you're done.

Comments can also be added to the margins of the compiled PDF using the todo command\todo{Here's a comment in the margin!}, as shown in the example on the right. You can also add inline comments:

\todo[inline, color=green!40]{This is an inline comment.}

\subsection{Lognormal}

\subsubsection{Parâmetros}

Use the table and tabular commands for basic tables --- see Table~\ref{tab:widgets}, for example. 

\begin{table}
\centering
\begin{tabular}{l|r}
Item & Quantity \\\hline
Widgets & 42 \\
Gadgets & 13
\end{tabular}
\caption{\label{tab:widgets}An example table.}
\end{table}

\subsection{Seleção de Modelos}

\LaTeX{} is great at typesetting mathematics. Let $X_1, X_2, \ldots, X_n$ be a sequence of independent and identically distributed random variables with $\text{E}[X_i] = \mu$ and $\text{Var}[X_i] = \sigma^2 < \infty$, and let
\[S_n = \frac{X_1 + X_2 + \cdots + X_n}{n}
      = \frac{1}{n}\sum_{i}^{n} X_i\]
denote their mean. Then as $n$ approaches infinity, the random variables $\sqrt{n}(S_n - \mu)$ converge in distribution to a normal $\mathcal{N}(0, \sigma^2)$.


\subsection{Aplicações}

Use section and subsections to organize your document. Simply use the section and subsection buttons in the toolbar to create them, and we'll handle all the formatting and numbering automatically.

\subsection{Modelo Final}

You can make lists with automatic numbering \dots

\begin{enumerate}
\item Like this,
\item and like this.
\end{enumerate}
\dots or bullet points \dots
\begin{itemize}
\item Like this,
\item and like this.
\end{itemize}

\subsection{Conclusão}

You can upload a file containing your BibTeX entries, created with JabRef; or import your, CiteULike or Zotero library as a file. You can then cite entries from it, like this. Just remember to specify a bibliography style, as well as the filename of the.

You can find a \href{https://www.overleaf.com/help/97-how-to-include-a-bibliography-using-bibtex}{video tutorial here} to learn more about BibTeX.

We hope you find Overleaf useful, and please let us know if you have any feedback using the help menu above --- or use the contact form at \url{https://www.overleaf.com/contact}!

\bibliographystyle{alpha}
\bibliography{Bibliografia}

\selectfont
\end{document}